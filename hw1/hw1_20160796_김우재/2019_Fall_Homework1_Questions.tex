%%%%%%%%%%%%%%%%%%%%%%%%%%%%%%%%%%%%%%%%%%%%%%%%%%%%%%%%%%%%%%%%%%%%%
%
% CS484 Written Question Template
%
% This is a LaTeX document. LaTeX is a markup language for producing 
% documents. Your task is to fill out this document, then to compile 
% it into a PDF document. 
%
% 
% TO COMPILE:
% > pdflatex thisfile.tex
%
% If you do not have LaTeX and need a LaTeX distribution:
% - Personal laptops (all common OS): www.latex-project.org/get/
% - We recommend miktex (https://miktex.org/) for latex engine,
%   and TeXstudio(http://www.texstudio.org/) for latex editor.
%   You should install both programs for editing latex.
%   Or you can use Overleaf (https://www.overleaf.com/) which is 
%   an online latex editor.
%
% If you need help with LaTeX, please come to office hours. 
% Or, there is plenty of help online:
% https://en.wikibooks.org/wiki/LaTeX
%
% Good luck!
% Min and the CS484 staff
%
%%%%%%%%%%%%%%%%%%%%%%%%%%%%%%%%%%%%%%%%%%%%%%%%%%%%%%%%%%%%%%%%%%%%%

\documentclass[11pt]{article}

\usepackage[english]{babel}
\usepackage[utf8]{inputenc}
\usepackage[colorlinks = true,
            linkcolor = blue,
            urlcolor  = blue]{hyperref}
\usepackage[a4paper,margin=1.5in]{geometry}
\usepackage{stackengine,graphicx}
\usepackage{fancyhdr}
\setlength{\headheight}{15pt}
\usepackage{microtype}
\usepackage{times}

% From https://ctan.org/pkg/matlab-prettifier
\usepackage[numbered,framed]{matlab-prettifier}

\frenchspacing
\setlength{\parindent}{0cm} % Default is 15pt.
\setlength{\parskip}{0.3cm plus1mm minus1mm}

\pagestyle{fancy}
\fancyhf{}
\lhead{Homework 1 Questions}
\rhead{CS484}
\rfoot{\thepage}

\date{}

\title{\vspace{-1cm}Homework 1 Questions}


\begin{document}
\maketitle
\vspace{-2cm}
\thispagestyle{fancy}

\section*{Instructions}
\begin{itemize}
  \item Compile and read through the included MATLAB tutorial.
  \item 2 questions.
  \item Include code.
  \item Feel free to include images or equations.
  \item Please make this document anonymous.
  \item \textbf{Please use only the space provided and keep the page breaks.} Please do not make new pages, nor remove pages. The document is a template to help grading.
  \item If you really need extra space, please use new pages at the end of the document and refer us to it in your answers.
\end{itemize}


\section*{Submission}
\begin{itemize}
	\item Please zip your folder with \textbf{hw1\_student id\_name.zip} $($ex: hw1\_20191234\_Peter.zip$)$
	\item Submit your homework to \href{http://klms.kaist.ac.kr/course/view.php?id=109597}{KLMS}.
	\item An assignment after its original due date will be degraded from the marked credit per day: e.g., A will be downgraded to B for one-day delayed submission.
\end{itemize}

\pagebreak


\section*{Questions}


%%%%%%%%%%%%%%%%%%%%%%%%%%%%%%%%%%%

% Please leave the pagebreak
\paragraph{Q1:} We wish to set all pixels that have a brightness of 10 or less to 0, to remove sensor noise. However, our code is slow when run on a database with 1000 grayscale images.

\emph{Image:} \href{grizzlypeakg.png}{grizzlypeakg.png}

\begin{lstlisting}[style=Matlab-editor]
A = imread('grizzlypeakg.png');
[m1,n1] = size( A );
for i=1:m1
    for j=1:n1
        if A(i,j) <= 10
            A(i,j) = 0;
        end
    end
end
\end{lstlisting}

\paragraph{Q1.1:} How could we speed it up?

%%%%%%%%%%%%%%%%%%%%%%%%%%%%%%%%%%%
\paragraph{A1.1:} We can speed up this sensor noise removal process by creating a binary logical array, B, instead of running a double loop of A. By running B = A $\le$ 10, B's elements will either be 0 or 1, depending on whether the pixel at corresponding indices of A is less than 10. Then, we can run A(B) = 0 to switch all pixels lower than 10 to 0. \\

\begin{lstlisting}[style=Matlab-editor]
A = imread('grizzlypeakg.png');
B = A <= 10;
A(B) = 0;
imshow(A)
\end{lstlisting}


%%%%%%%%%%%%%%%%%%%%%%%%%%%%%%%%%%%

% Please leave the pagebreak
\pagebreak
\paragraph{Q1.2:} What factor speedup would we receive over 1000 images? Please measure it.

Ignore file loading; assume all images are equal resolution; don't assume that the time taken for one image $\times1000$ will equal $1000$ image computations, as single short tasks on multitasking computers often take variable time.

%%%%%%%%%%%%%%%%%%%%%%%%%%%%%%%%%%%
\paragraph{A1.2:} In order to measure time it takes for the previous and revised versions of sensor removal, I ran two .m files, each containing previous and revised version, 1000 times using a for loop in a separate .m file. \\
As a result, the original method of using a double loop took $55.479265$ seconds, and the revised method using a binary logical array took $41.968613$ seconds. Thus, the factor speedup is $55.479265/41.968613 = \framebox{1.3219228}$. \\

Running original algorithm 1000 times:
\begin{lstlisting} [style=Matlab-editor]
tic;
for i1 = 1:1000
    Q1_1_original
end
toc;
\end{lstlisting}

Running revised algorithm 1000 times:
\begin{lstlisting} [style=Matlab-editor]
tic;
for i1 = 1:1000
    Q1_1_revised
end
toc;
\end{lstlisting}


%%%%%%%%%%%%%%%%%%%%%%%%%%%%%%%%%%%

% Please leave the pagebreak
\pagebreak
\paragraph{Q1.3:} How might a speeded-up version change for color images? Please measure it.

\emph{Image:} \href{grizzlypeak.jpg}{grizzlypeak.jpg}

%%%%%%%%%%%%%%%%%%%%%%%%%%%%%%%%%%%
\paragraph{A1.3:} When run single time, the original method took $8.268699$ seconds, and the revised method took $0.107157$ second. The gap between the two execution times using colored picture is much greater than the one using previous picture - with the speedup factor of about $77.164338$. This might be so because the original method loops around a lot more pixels consisting of red, green, and blue componenets, whereas the revised method finds pixels less than 10 using a logical binary array without the need to loop through the picture's pixels.

Original algorithm:
\begin{lstlisting} [style=Matlab-editor]
tic;
A = imread('grizzlypeak.jpg');
[m1, n1] = size(A);
for i = 1: m1
    for j = 1 : n1
        if A(i, j) <= 10
            A(i, j) = 0;
        end
    end
end
toc;
imshow(A) 
\end{lstlisting}

Revised algorithm:
\begin{lstlisting} [style=Matlab-editor]
tic;
A = imread('grizzlypeak.jpg');
B = A <= 10;
A(B) = 0;
toc;
imshow(A)
\end{lstlisting}



%%%%%%%%%%%%%%%%%%%%%%%%%%%%%%%%%%%

% Please leave the pagebreak
\pagebreak
\paragraph{Q2:} We wish to reduce the brightness of an image but, when trying to visualize the result, all we sees is white with some weird ``corruption'' of color patches.

\emph{Image:} \href{gigi.jpg}{gigi.jpg}

\begin{lstlisting}[style=Matlab-editor]
I = double( imread('gigi.jpg') );
I = I - 20;
imshow( I );
\end{lstlisting}

\paragraph{Q2.1:} What is incorrect with this approach? How can it be fixed while maintaining the same amount of brightness reduction?

%%%%%%%%%%%%%%%%%%%%%%%%%%%%%%%%%%%
\paragraph{A2.1:} \mbox{} \\
1) In double form, all numbers that exceed 1.0 are considered as 1. Therefore, all pixels that exceed 1 have the same brightness - white. We can fix this problem by normalizing pixels to lie between 0 and 1.0. I did so by changing double to im2double. \\
2) To reduce the brightness, the original code subtracts 20 from all elements of I. However, the elements are now normalized to lie between 0 and 1.0. Therefore, to reduce the brightness by same amount, we need to reduce pixels by 20/255, not 20.

\begin{lstlisting} [style=Matlab-editor]
I = im2double( imread('gigi.jpg') );
I = I - 20/255;
imshow( I );
\end{lstlisting}



%%%%%%%%%%%%%%%%%%%%%%%%%%%%%%%%%%%

% Please leave the pagebreak
\pagebreak
\paragraph{Q2.2:} Where did the original corruption come from? Which specific values in the original image did it represent?

%%%%%%%%%%%%%%%%%%%%%%%%%%%%%%%%%%%
\paragraph{A2.2:} When run with original method, the result showed a white picture with some yellowish patches - corruption. When the image is in double form, it considers all pixels higher than 1.0 as 1 - the maximum brightness - and prints them as a white pixel. Thus, yellowish patches represent pixels that are lower than 1.0.

\emph{Fixed image:} \href{fixedgigi.jpg}{fixedgigi.jpg}


%%%%%%%%%%%%%%%%%%%%%%%%%%%%%%%%%%%

\end{document}
