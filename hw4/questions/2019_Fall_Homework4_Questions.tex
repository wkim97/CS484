%%%%%%%%%%%%%%%%%%%%%%%%%%%%%%%%%%%%%%%%%%%%%%%%%%%%%%%%%%%%%%%%%%%%%%%%%%%%%%%%%%%%%%%%%%%%%%%%
%
% CS484 Written Question Template
%
% Acknowledgements:
% The original code is written by Prof. James Tompkin (james_tompkin@brown.edu).
% The second version is revised by Prof. Min H. Kim (minhkim@kaist.ac.kr).
%
% This is a LaTeX document. LaTeX is a markup language for producing 
% documents. Your task is to fill out this document, then to compile 
% it into a PDF document. 
%
% 
% TO COMPILE:
% > pdflatex thisfile.tex
%
% If you do not have LaTeX and need a LaTeX distribution:
% - Personal laptops (all common OS): www.latex-project.org/get/
% - We recommend latex compiler miktex (https://miktex.org/) for windows,
%   macTex (http://www.tug.org/mactex/) for macOS users.
%   And TeXstudio(http://www.texstudio.org/) for latex editor.
%   You should install both compiler and editor for editing latex.
%   The another option is Overleaf (https://www.overleaf.com/) which is 
%   an online latex editor.
%
% If you need help with LaTeX, please come to office hours. 
% Or, there is plenty of help online:
% https://en.wikibooks.org/wiki/LaTeX
%
% Good luck!
% Min and the CS484 staff
%
%%%%%%%%%%%%%%%%%%%%%%%%%%%%%%%%%%%%%%%%%%%%%%%%%%%%%%%%%%%%%%%%%%%%%%%%%%%%%%%%%%%%%%%%%%%%%%%%
%
% How to include two graphics on the same line:
% 
% \includegraphics[width=0.49\linewidth]{yourgraphic1.png}
% \includegraphics[width=0.49\linewidth]{yourgraphic2.png}
%
% How to include equations:
%
% \begin{equation}
% y = mx+c
% \end{equation}
% 
%%%%%%%%%%%%%%%%%%%%%%%%%%%%%%%%%%%%%%%%%%%%%%%%%%%%%%%%%%%%%%%%%%%%%%%%%%%%%%%%%%%%%%%%%%%%%%%%

\documentclass[11pt]{article}

\usepackage[english]{babel}
\usepackage[utf8]{inputenc}
\usepackage[colorlinks = true,
            linkcolor = blue,
            urlcolor  = blue]{hyperref}
\usepackage[a4paper,margin=1.5in]{geometry}
\usepackage{stackengine,graphicx}
\usepackage{fancyhdr}
\setlength{\headheight}{15pt}
\usepackage{microtype}
\usepackage{times}
\usepackage{amsmath}
\usepackage{subcaption}

% From https://ctan.org/pkg/matlab-prettifier
\usepackage[numbered,framed]{matlab-prettifier}

\frenchspacing
\setlength{\parindent}{0cm} % Default is 15pt.
\setlength{\parskip}{0.3cm plus1mm minus1mm}

\pagestyle{fancy}
\fancyhf{}
\lhead{Homework 4 Questions}
\rhead{CS 484}
\rfoot{\thepage}

\date{}

\title{\vspace{-1cm}Homework 4 Questions}


\begin{document}
\maketitle
\vspace{-3cm}
\thispagestyle{fancy}

\section*{Instructions}
\begin{itemize}
  \item 4 questions.
  \item Write code where appropriate.
  \item Feel free to include images or equations.
  \item Please make this document anonymous.
  \item \textbf{Please use only the space provided and keep the page breaks.} Please do not make new pages, nor remove pages. The document is a template to help grading.
  \item If you really need extra space, please use new pages at the end of the document and refer us to it in your answers.
\end{itemize}

\section*{Questions}

\paragraph{Q1:} Imagine we were tasked with designing a feature point which could match all of the following three pairs of images. Which real world phenomena and camera effects might cause us problems?
Use the MATLAB function \href{https://www.mathworks.com/help/images/ref/corner.html}{$corner$} to investigate. $corner(I,1000)$.

\emph{RISHLibrary} | \emph{Chase} | \emph{LaddObservatory}

%%%%%%%%%%%%%%%%%%%%%%%%%%%%%%%%%%%
\paragraph{A1:} 
From RISHLibrary, we can see that feature points are invariant to illumination. However, shift in camera angle introduces new objects to the image and thus causes feature points to change. As shown in Chase images, blur also causes feature points to change, but not much. Most of the corners are correctly detected in Chase1.jpg and Chase2.jpg with exception of a few different ones. LaddObservatory images show that feature points are not invariant to zooming in of camera. Detected feature points are different in differently zoomed versions of LaddObservatory images.
\begin{lstlisting}[style=Matlab-editor, basicstyle=\small]
image = imread('Chase1.jpg');
g_image = rgb2gray(image);
C = corner(g_image, 1000);
\end{lstlisting}
\begin{figure} [!b]
	\centering
	\begin{subfigure}{0.16\linewidth}
		\includegraphics[width=\linewidth]{eval_RISHLibrary1.png}
	\end{subfigure} 
	% 
	\begin{subfigure}{0.16\linewidth}
		\includegraphics[width=\linewidth]{eval_RISHLibrary2.png}
	\end{subfigure}
	%
	\begin{subfigure}{0.16\linewidth}
		\includegraphics[width=\linewidth]{eval_Chase1.png}
	\end{subfigure}
	%
	\begin{subfigure}{0.16\linewidth}
		\includegraphics[width=\linewidth]{eval_Chase2.png}
	\end{subfigure}
	%
	\begin{subfigure}{0.16\linewidth}
		\includegraphics[width=\linewidth]{eval_Ladd1.png}
	\end{subfigure}
	%
	\begin{subfigure}{0.16\linewidth}
		\includegraphics[width=\linewidth]{eval_Ladd2.png}
	\end{subfigure}
	%
\end{figure}




%%%%%%%%%%%%%%%%%%%%%%%%%%%%%%%%%%%

% Please leave the pagebreak
\pagebreak
\paragraph{Q2:} In designing our feature point, what characteristics might we wish it to have? Describe the fundamental trade-off between feature point invariance and discriminative power. How should we design for this trade-off?

%%%%%%%%%%%%%%%%%%%%%%%%%%%%%%%%%%%
\paragraph{A2:} \mbox{} \\ 
In designing our feature point, we wish it to be invariant to be accidental image transformations, including both apperance variance (difference in brightness, illumination, etc) and geometric variation (camera rotation, camera translation, etc). However, at the same time, we also wish it to sustain discriminative power, or ability to actually determine the difference between two distinct features.\\

A good image descriptor should be largely invariant to such transformations. However, a feature that is perfectly invariant has little or no discriminative power. Similarly, a point with high discriminative power has low invariance. This is so because as a feature point invariance increases, the ability for it to distinguish two different points diminishes. For example, descriptors that are rotationally invariant have poor discriminative power - they will map patches that look different to the same descriptor. The problem thus lies on finding the trade-off between feature point invariance and discriminative power. \\

To find a point at which a point can satisfy both invariance and discriminative power, we can first start off with a perfectly invariant point. Then, we can decrease its invariance little by little until discriminative power that is satisfory is achieved.


%%%%%%%%%%%%%%%%%%%%%%%%%%%%%%%%%%%

% Please leave the pagebreak
\pagebreak
\paragraph{Q3:} In the Harris corner detector, what do the eigenvalues of the `M' second moment matrix represent? Discuss both how they relate to image intensity and how we can interpret them geometrically.

%%%%%%%%%%%%%%%%%%%%%%%%%%%%%%%%%%%
\paragraph{A3:} \mbox{} \\
Second moment matrix 'M' as given as: \\
$M = \sum_{x,y} w(x,y) \begin{bmatrix} I_{x}^2 & I_{x}I_{y} \\ I_{x}I_{y} & I_{y}^2 \end{bmatrix} $.\\

M can also be given as a diagonalized form as: \\
$M = R^{-1} \begin{bmatrix} \lambda_1 & 0 \\ 0 & \lambda_2 \end{bmatrix} R$. \\

Consider an ellipse that is formed when $E(u,v) = \begin{bmatrix} u & v \end{bmatrix} M \begin{bmatrix} u \\ v \end{bmatrix}$ is cut by a horizontal slice. Eigenvalues $\lambda_1$ and $\lambda_2$ of M determined the axis lengths of formed ellipse, which are given as $\dfrac{1}{\sqrt{\lambda_{max}}}$ and $\dfrac{1}{\sqrt{\lambda_{min}}}$. \\ Thus, we can also see that eigenvalues determine whether the region is corner or not. If either one of the eigenvalues is approximately 0, the region is not a corner. \\

Since eigenvalues show if the region is a corner, edge, or a flat region, it also tells us if the region contains large intensity changes or not. If both eigenvalues are not close to zero, then the region contains a corner, and shifting a window will give large intensity changes. If one eigenvalue is close to zero, then the region contains an edge, and shifting a window in only a certain direction will give intensity change. If both eigenvalues are near zero, then the region is flat, meaning there are little intensity changes in that region.


%%%%%%%%%%%%%%%%%%%%%%%%%%%%%%%%%%%


% Please leave the pagebreak
\pagebreak
\paragraph{Q4:} Explain the difference between the Euclidean distance and the cosine similarity metrics between descriptors. What might their geometric interpretations reveal about when each should be used? Given a distance metric, what is a good method for feature descriptor matching and why?

%%%%%%%%%%%%%%%%%%%%%%%%%%%%%%%%%%%
\paragraph{A4:} \mbox{} \\
For two vectors $P = (p_1, p_2, ..., p_n)$ and $Q = (q_1, q_2, ..., q_n)$, \\

Euclidean distance looks at the distance between the two vectors. It is given as:\\ $\sqrt{(p_1 - q_1)^2 + (p_2 - q_2)^2 + ... + (p_n - q_n)^2} = \sqrt{\sum_{i=1}^{n}(x_i - y_i)^2}$.\\

Cosine similarity looks at the angle between the two vectors. It is used when the magnitudes of vectors do not matter. It is given as:\\ $\text{similarity} = \dfrac{P \cdot Q}{\| P \|_2 \| Q \|_2}$, where $a \cdot b = \| a \|_2 \| b \|_2 \cos \theta$. \\

Their geometric interpretations imply that cosine similarity is better than Euclidean distance when dealing with angles between the two vectors, since it computes the inner product of two vectors. Cosine similiarity is not affected by the magnitude of vectors. \\

In constrast, Euclidean distance is better than cosine similarity when the magnitudes of vectors need to be taken into consideration. It is used when the angle between the two vectors is not as important as the actual distance between them. \\

For feature descriptor matching, the feature descriptors are often not normalized. Therefore, it would be better to use cosine similarity method. Two similar features could be close together in terms of angle, but could be far away from each other in terms of magnitude. In that case, Euclidean distance method will determine those two vectors to be as far away from each other, whereas cosine similarity will determine them to be similar. In cases of feature descriptor matching, where the magnitudes of vectors do not matter, cosine similarity is the better method. 




%%%%%%%%%%%%%%%%%%%%%%%%%%%%%%%%%%%


% If you really need extra space, uncomment here and use extra pages after the last question.
% Please refer here in your original answer. Thanks!
%\pagebreak
%\paragraph{AX.X Continued:} Your answer continued here.



\end{document}
