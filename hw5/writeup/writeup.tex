%%%%%%%%%%%%%%%%%%%%%%%%%%%%%%%%%%%%%%%%%%%%%%%%%%%%%%%%%%%%%%%%%%%%%%%%%%%%%%%%%%%%%%%%%%%%%%%%
%
% CS484 Written Question Template
%
% Acknowledgements:
% The original code is written by Prof. James Tompkin (james_tompkin@brown.edu).
% The second version is revised by Prof. Min H. Kim (minhkim@kaist.ac.kr).
%
% This is a LaTeX document. LaTeX is a markup language for producing 
% documents. Your task is to fill out this document, then to compile 
% it into a PDF document. 
%
% 
% TO COMPILE:
% > pdflatex thisfile.tex
%
% If you do not have LaTeX and need a LaTeX distribution:
% - Personal laptops (all common OS): www.latex-project.org/get/
% - We recommend latex compiler miktex (https://miktex.org/) for windows,
%   macTex (http://www.tug.org/mactex/) for macOS users.
%   And TeXstudio(http://www.texstudio.org/) for latex editor.
%   You should install both compiler and editor for editing latex.
%   The another option is Overleaf (https://www.overleaf.com/) which is 
%   an online latex editor.
%
% If you need help with LaTeX, please come to office hours. 
% Or, there is plenty of help online:
% https://en.wikibooks.org/wiki/LaTeX
%
% Good luck!
% Min and the CS484 staff
%
%%%%%%%%%%%%%%%%%%%%%%%%%%%%%%%%%%%%%%%%%%%%%%%%%%%%%%%%%%%%%%%%%%%%%%%%%%%%%%%%%%%%%%%%%%%%%%%%
%
% How to include two graphics on the same line:
% 
% \includegraphics[\width=0.49\linewidth]{yourgraphic1.png}
% \includegraphics[\width=0.49\linewidth]{yourgraphic2.png}
%
% How to include equations:
%
% \begin{equation}
% y = mx+c
% \end{equation}
% 
%%%%%%%%%%%%%%%%%%%%%%%%%%%%%%%%%%%%%%%%%%%%%%%%%%%%%%%%%%%%%%%%%%%%%%%%%%%%%%%%%%%%%%%%%%%%%%%%

\documentclass[11pt]{article}

\usepackage[english]{babel}
\usepackage[utf8]{inputenc}
\usepackage[colorlinks = true,
linkcolor = blue,
urlcolor  = blue]{hyperref}
\usepackage[a4paper,margin=1.5in]{geometry}
\usepackage{stackengine,graphicx}
\usepackage{fancyhdr}
\setlength{\headheight}{15pt}
\usepackage{microtype}
\usepackage{times}
\usepackage{booktabs}
\usepackage{amsmath}
\usepackage{subcaption}
\usepackage{wrapfig}


% From https://ctan.org/pkg/matlab-prettifier
\usepackage[numbered,framed]{matlab-prettifier}

\frenchspacing
\setlength{\parindent}{0cm} % Default is 15pt.
\setlength{\parskip}{0.3cm plus1mm minus1mm}

\pagestyle{fancy}
\fancyhf{}
\lhead{Homework Writeup}
\rhead{CS484}
\rfoot{\thepage}

\date{}

\title{\vspace{-1cm}Homework 5 Writeup}


\begin{document}
\maketitle
\vspace{-3cm}
\thispagestyle{fancy}

\section*{Instructions}
\begin{itemize}
  \item Describe any interesting decisions you made to write your algorithm.
  \item Show and discuss the results of your algorithm.
  \item Feel free to include code snippets, images, and equations.
  \item Use as many pages as you need, but err on the short side If you feel you only need to write a short amount to meet the brief, th
  
  \item \textbf{Please make this document anonymous.}
\end{itemize}

\section*{Implementation and Algorithms}
\subsection*{Tiny Images}
The goal of the function \textbf{get$\_$tiny$\_$images.m} is to produce a feature representation for an image using a "tiny image" method. This method simply resizes the image to a smaller one. I first resized the image into 16x16 pixels. Then, I zero-centered the image and normalized it by dividing it by standard deviation of the image. The code snippet is shown below:
\begin{lstlisting}[style=Matlab-editor]
[N,~] = size(image_paths);
image_feats = zeros(N,256);
for i = 1:N
    image_path = image_paths{i};
    image = imread(image_path);
    resized_image = double(imresize(image, [16,16]));
    resized_image = resized_image';
    resized_image = resized_image(:)';
    resized_image = resized_image - mean(resized_image);
    resized_image = resized_image ./ norm(resized_image);
    image_feats(i,:) = resized_image;
end
\end{lstlisting}

\subsection*{Nearest Neighbor}
The goal of the function \textbf{nearest$\_$neighbor$\_$classify.m} is to produce a classifier by using nearest neighbor method. It takes the training image features, training labels, and test image features as parameters and find the distance between all the features between training image features and test image features. Then, it selects K (I chose 15 for K) nearest points and selects the most frequently occurring label as the test feature's label. The code snippet is shown below:
\begin{lstlisting}[style=Matlab-editor]
K = 15;
[M,~] = size(test_image_feats);
[N,~] = size(train_image_feats);
predicted_categories = cell(M,1);
for i=1:M
    labels = train_labels;
    dist = zeros(N,1);
    test_image = test_image_feats(i,:);
    dist = pdist2(train_image_feats, test_image);
    [dist,ind] = sort(dist);
    l = labels;
    labels = labels(ind);
    candidate_labels = labels(1:K);
    label = string(mode(categorical(candidate_labels)));
    predicted_categories{i} = label;
end
\end{lstlisting}

\subsection*{Bag of Words}
Bag of words method is a feature representation method that extracts features from images and stores them into a bag. \\

The goal of the function \textbf{build$\_$vocabulary.m} is to produce histograms of features and to group all features of all training images into 200 clusters. In order to produce histograms, I used the matlab function \textbf{extractHOGFeatures}. Then, I clustered them into vocab$\_$size groups by using K-mean clustering. I used the matlab function \textbf{kmeans} in order to do so. The code snippet is shown below: 
\begin{lstlisting}[style=Matlab-editor]
[N,~] = size(image_paths);
bag_features = [];
for i =1:N
    image_path = image_paths{i};
    img_features = [];
    image = imread(image_path);
    [hog,~] = extractHOGFeatures(image,'CellSize',[16 16]);
    [~,hog_size] = size(hog);
    num_blocks = hog_size / 36;
    for b = 1:num_blocks
        img_features = [img_features; hog(36*(b-1)+1:36*b)];
    end
    bag_features = [bag_features; img_features];
end
[~, vocab] = kmeans(bag_features, vocab_size);
\end{lstlisting}
The goal of the function \textbf{get$\_$bags$\_$of$\_$words.m} is to get the $vocab\_size \times D$ codebook matrix of codevectors produced by \textbf{build$\_$vocabulary.m} and determine how frequently each codevector occurs in the given image. This is done for all images in the image$\_$paths, and the function returns matrix of N images, each with d-dimensional feature representation vector. In this case, d is equal to the number of clusters, each with relative frequency of each codevector for the specific image. The code snippet is shown below:
\begin{lstlisting}[style=Matlab-editor]
load('vocab.mat')
vocab_size = size(vocab, 1);
[N,~] = size(image_paths);
image_feats = zeros(N,vocab_size);
for i =1:N
    image_path = image_paths{i};
    image = imread(image_path);
    [hog,~] = extractHOGFeatures(image,'CellSize',[16 16]);
    [~,hog_size] = size(hog);
    num_blocks = hog_size / 36;
    for b = 1:num_blocks
        feature = hog(36*(b-1)+1:36*b);
        dist = pdist2(vocab, feature);
        [dist, ind] = sort(dist);
        image_feats(i,ind(1)) = image_feats(i,ind(1)) + 1;
    end
    image_feats(i,:) = image_feats(i,:) ./ norm(image_feats(i,:));
end
\end{lstlisting}

\subsection*{SVM}
SVM, or supporting vector machine, is a linear classifier that determines whether a feature belongs to the positive side or the negative side. The goal of the function \textbf{svm$\_$classify.m} is to predict the label for an image given the training data and labels by using multiple binary SVMs. Since SVM is a binary classifier, and because there are 15 classifiers, I repeated the modeling 15 times by categorizing each into binary classifiers, or "forest" vs "non-forest", and "kitchen" vs "non-kitchen", for example. I used the Matlab function \textbf{fitcsvm} in order to obtain the SVM model for each classification. To predict the label for test image, I repeated the fit with 15 models for each classifier, and I determined the best label, or the label with highest fit. The code snippet for training is shown below:
\begin{lstlisting}[style=Matlab-editor]
for c = 1:num_categories
    inds = zeros(N,1);
    inds(:,:) = -1;
    inds(strcmp(train_labels, categories{c})) = 1;
    model = fitcsvm(train_image_feats, inds);
    B = model.Bias;
    W = (model.Beta)';
    B_results = [B_results; B];
    W_results = [W_results; W];
end
\end{lstlisting}
The code snippet for testing is shown below:
\begin{lstlisting}[style=Matlab-editor]
for i = 1:M
    best_label = zeros(num_categories,1);
    for l = 1:num_categories
        best_label(l) = dot(W_results(l,:), test_image_feats(i,:)) + B_results(l);
    end
    [best_label,ind] = sort(best_label, 'descend');
    predicted_categories{i} = categories{ind(1)};
end
\end{lstlisting}

\section*{Performance}
As shown in Table 1, performance, or the accuracy, increases as new method is implemented. Bag of words method performs as a better feature representation of image than tiny image method does, and SVM method performs as a better classifier than nearest neighbor method does. Particularly, SVM produces better results in a faster time than nearest neighbor does.

\begin{table}[!ht]
	\begin{center}

		\begin{tabular}{|c| c| c| c|}  
			\hline
			
			Feature & Classifier & Accuracy & Time (in sec) \\ [0.5ex] 
			\hline \hline
			placeholder & placeholder  &0.067  & 2.516878  \\ 
			\hline
			tiny image & nearest neighbor  &0.221  & 31.013444  \\ 
			\hline
			bag of words & nearest neighbor  &0.491  & 180.424759 \\
			\hline
			bag of words & SVM  & 0.584 & 163.632035 \\
			\hline
		\end{tabular}
		\caption{Performance metrics for different pipelines} \label{pos1}
	\end{center}
\end{table}
As shown in the table, our best recognition setup was using "bag of words" feature representation and "SVM" classifier.

\newpage
\section*{Results}
Using bag of words as feature descriptor representation and SVM as classifier produces the highest accuracy. Full confusion matrix and table of classification decisions are shown below.

\begin{figure}[!ht]
	\centering
	\includegraphics[width=\linewidth]{../results/bog_svm.png}
	\caption{Full confusion map \\ Feature representation: bag of words, Classifier: SVM}
\end{figure}

\begin{figure}[!ht]
	\centering
	\includegraphics[width=\linewidth]{../results/table_bog_svm.png}
	\caption{Table of classification decisions \\ Feature representation: bag of words, Classifier: SVM}
\end{figure}

\end{document}
