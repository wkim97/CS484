%%%%%%%%%%%%%%%%%%%%%%%%%%%%%%%%%%%%%%%%%%%%%%%%%%%%%%%%%%%%%%%%%%%%%%%%%%%%%%%%%%%%%%%%%%%%%%%%
%
% CS484 Written Question Template
%
% Acknowledgements:
% The original code is written by Prof. James Tompkin (james_tompkin@brown.edu).
% The second version is revised by Prof. Min H. Kim (minhkim@kaist.ac.kr).
%
% This is a LaTeX document. LaTeX is a markup language for producing 
% documents. Your task is to fill out this document, then to compile 
% it into a PDF document. 
%
% 
% TO COMPILE:
% > pdflatex thisfile.tex
%
% If you do not have LaTeX and need a LaTeX distribution:
% - Personal laptops (all common OS): www.latex-project.org/get/
% - We recommend latex compiler miktex (https://miktex.org/) for windows,
%   macTex (http://www.tug.org/mactex/) for macOS users.
%   And TeXstudio(http://www.texstudio.org/) for latex editor.
%   You should install both compiler and editor for editing latex.
%   The another option is Overleaf (https://www.overleaf.com/) which is 
%   an online latex editor.
%
% If you need help with LaTeX, please come to office hours. 
% Or, there is plenty of help online:
% https://en.wikibooks.org/wiki/LaTeX
%
% Good luck!
% Min and the CS484 staff
%
%%%%%%%%%%%%%%%%%%%%%%%%%%%%%%%%%%%%%%%%%%%%%%%%%%%%%%%%%%%%%%%%%%%%%%%%%%%%%%%%%%%%%%%%%%%%%%%%
%
% How to include two graphics on the same line:
% 
% \includegraphics[width=0.49\linewidth]{yourgraphic1.png}
% \includegraphics[width=0.49\linewidth]{yourgraphic2.png}
%
% How to include equations:
%
% \begin{equation}
% y = mx+c
% \end{equation}
% 
%%%%%%%%%%%%%%%%%%%%%%%%%%%%%%%%%%%%%%%%%%%%%%%%%%%%%%%%%%%%%%%%%%%%%%%%%%%%%%%%%%%%%%%%%%%%%%%%

\documentclass[11pt]{article}

\usepackage[english]{babel}
\usepackage[utf8]{inputenc}
\usepackage[colorlinks = true,
            linkcolor = blue,
            urlcolor  = blue]{hyperref}
\usepackage[a4paper,margin=1.5in]{geometry}
\usepackage{stackengine,graphicx}
\usepackage{fancyhdr}
\setlength{\headheight}{15pt}
\usepackage{microtype}
\usepackage{times}
\usepackage{booktabs}
\usepackage{subcaption}
\captionsetup{justification=raggedright,singlelinecheck=false}
\usepackage{indentfirst}

% From https://ctan.org/pkg/matlab-prettifier
\usepackage[numbered,framed]{matlab-prettifier}

\frenchspacing
\setlength{\parindent}{0cm} % Default is 15pt.
\setlength{\parskip}{0.3cm plus1mm minus1mm}

\pagestyle{fancy}
\fancyhf{}
\lhead{Homework Writeup}
\rhead{CS 484}
\rfoot{\thepage}

\date{}

\title{\vspace{-1cm}Homework 2 Writeup}


\begin{document}
\maketitle
\vspace{-3cm}
\thispagestyle{fancy}

\section*{Instructions}
\begin{itemize}
  \item Describe any interesting decisions you made to write your algorithm.
  \item Show and discuss the results of your algorithm.
  \item Feel free to include code snippets, images, and equations.
  \item Use as many pages as you need, but err on the short side If you feel you only need to write a short amount to meet the brief, th
  
  \item \textbf{Please make this document anonymous.}
\end{itemize}

\section*{Image Filtering} 
I used Fast Fourier Transformation to implement image filtering. \par
First, I padded the image with zeros according to the size of filter. For top and bottom of the pixel matrix, "floor(filter$\_$size(1) / 2)" rows of 0 were added, and for right and left of the pixel matrix, "floor(filter$\_$size(2) / 2)" columns of 0 were added (filter$\_$size = size(filter)). I also padded the filter using the same size and put the actual filter on the upper left corner.\par
Then, I ran a loop k many times, where dimension of the image is m $\times$ n $\times$ k. So, if the image is grayscale, the loop is run 1 time, and if the image is colored, the loop if run 3 times. Inside the loop, I found Fast Fourier Transformation of padded image and padded filter using fft2 function. I then found padded convolution by multiplying elements of padded image and padded filter. I then used ifft2 function to find image of that product and found appropriate portion to unpad the image. \par
Shown below is the code snippet: \par
\begin{lstlisting}[style=Matlab-editor]
filter_size = size(filter);
if rem(filter_size(1), 2) == 0 || rem(filter_size(2), 2) == 0
    error("Error: even-sized filter!");
end

filter_row = floor(filter_size(1)/2);
filter_col = floor(filter_size(2)/2);
image_size = size(image);

pimage = padarray(image, [filter_row, filter_col]);
pimage_size = size(pimage);
pfilter = zeros(pimage_size(1), pimage_size(2));
pfilter(1:filter_size(1), 1:filter_size(2)) = filter;
ret = zeros(image_size);

for k = 1:image_size(3)
    f_image = fft2(pimage(:,:,k));
    f_filter = fft2(pfilter);
    padded_conv = ifft2(f_image .* f_filter);
    ret(:,:,k) = padded_conv(2 * filter_row + 1 : pimage_size(1), 2 * filter_col + 1 : pimage_size(2));
end 
output = ret;
\end{lstlisting}

Shown below are images that resulted from this code:
\begin{figure} [!b]
	\centering
	\begin{subfigure}{0.3\linewidth}
		\includegraphics[width=\linewidth]{cat.jpg}
		\caption{Original picture:\\ 
		\href{cat.jpg}{cat.jpg}}
	\end{subfigure} \\
	% 
	\begin{subfigure}{0.3\linewidth}
		\includegraphics[width=\linewidth]{identity_image.jpg}
		\caption{Identity filter:\\ \href{identity_image.jpg}{identity$\_$image.jpg}}
	\end{subfigure}
	%
	\begin{subfigure}{0.3\linewidth}
		\includegraphics[width=\linewidth]{blur_image.jpg}
		\caption{Small blur with box filter:\\ \href{blur_image.jpg}{blur$\_$image.jpg}}
	\end{subfigure}
	%
	\begin{subfigure}{0.3\linewidth}
		\includegraphics[width=\linewidth]{large_blur_image.jpg}
		\caption{Large blur:\\ \href{large_blur_image.jpg}{large$\_$blur$\_$image.jpg}}
	\end{subfigure}
	%
	\begin{subfigure}{0.3\linewidth}
		\includegraphics[width=\linewidth]{sobel_image.jpg}
		\caption{Oriented filter (Sobel Operator):\\ \href{sobel_image.jpg}{sobel$\_$image.jpg}}
	\end{subfigure}
	%
	\begin{subfigure}{0.3\linewidth}
		\includegraphics[width=\linewidth]{laplacian_image.jpg}
		\caption{High pass filter (Discrete Laplacian):\\ \href{laplacian_image.jpg}{laplacian$\_$image.jpg}}
	\end{subfigure}
	%
	\begin{subfigure}{0.3\linewidth}
		\includegraphics[width=\linewidth]{high_pass_image.jpg}
		\caption{High pass "filter" alternative:\\ \href{high_pass_image.jpg}{high$\_$pass$\_$image.jpg}}
	\end{subfigure}
	%
\end{figure}

\section*{Hybrid Image}

For part 2, I implemented the function gen$\_$hybrid$\_$image(image1, image2, cutoff$\_$frequency). This function first filters image1 through a low pass filter and image2 through a high pass filter. Then, it adds those two outputs up to create a new hybrid image. \par
First, I created a square n $\times$ n Gaussian filter, where n is 2 $\times$ cutoff$\_$frequency $+$ 1, and standard deviation is cutoff$\_$frequency. Then, low$\_$frequencies was calculated by filtering image1 through the filter, creating a blurred version of image1. Likewise, high$\_$frequencies was calculated by subtracting blurred version of image2 - that was created by filtering image2 with filter - from image2 itself. Finally, I simply added those two matrices up to get hybrid$\_$image. \par
Shown below is the code snippet:
\begin{lstlisting}[style=Matlab-editor]
%%%%%%%%%%%%%%%%%%%%%%%%%%%%%%%%%%%%%%%%%%%%%%%%%%%%%%%%
% Remove high frequency to produce low frequencies only
%%%%%%%%%%%%%%%%%%%%%%%%%%%%%%%%%%%%%%%%%%%%%%%%%%%%%%%%
filter = fspecial('Gaussian', cutoff_frequency*2+1, cutoff_frequency);
high_pass_image = my_imfilter(image1, filter);
high_pass_image = my_imfilter(high_pass_image, filter');
low_frequencies = [high_pass_image];

%%%%%%%%%%%%%%%%%%%%%%%%%%%%%%%%%%%%%%%%%%%%%%%%%%%%%%%%
% Remove low frequency to produce high frequencies only
%%%%%%%%%%%%%%%%%%%%%%%%%%%%%%%%%%%%%%%%%%%%%%%%%%%%%%%%
low_pass_image = my_imfilter(image2, filter);
low_pass_image = my_imfilter(low_pass_image, filter');
low_pass_image = image2 - low_pass_image;
high_frequencies = [low_pass_image];

%%%%%%%%%%%%%%%%%%%%%%%%%%%%%%%%%%%%%%%%%%%%%%%%%%%%%%%%
% Combine the high frequencies and low frequencies
%%%%%%%%%%%%%%%%%%%%%%%%%%%%%%%%%%%%%%%%%%%%%%%%%%%%%%%%
hybrid_image = [high_pass_image + low_pass_image];
\end{lstlisting}

\newpage
Shown below are combinations of image1 = cat and image2 = dog:
\begin{figure} [!htb]
	\begin{subfigure}{0.5\textwidth}
		\centering
		\includegraphics[width=\textwidth]{dog.jpg}
		\caption{Image1:\\ \href{dog.jpg}{dog.jpg}}
	\end{subfigure}
	%
	\begin{subfigure}{0.5\textwidth}
		\centering
		\includegraphics[width=\textwidth]{low_frequencies.jpg}
		\caption{Image1 low frequencies:\\ \href{low_frequencies.jpg}{low$\_$frequencies.jpg}}
	\end{subfigure}
	%
	\begin{subfigure}{0.5\textwidth}
		\centering
		\includegraphics[width=\textwidth]{cat.jpg}
		\caption{Image2:\\ \href{cat.jpg}{cat.jpg}}
	\end{subfigure}
	%
	\begin{subfigure}{0.5\textwidth}
		\centering
		\includegraphics[width=\textwidth]{high_frequencies.jpg}
		\caption{Image2 high frequencies:\\ \href{high_frequencies.jpg}{high$\_$frequencies.jpg}}
	\end{subfigure}
	%
	\begin{subfigure}{0.5\textwidth}
		\centering
		\includegraphics[width=\textwidth]{hybrid_image.jpg}
		\caption{Two image combined:\\ \href{hybrid_image.jpg}{hybrid$\_$image.jpg}}
	\end{subfigure}
	%
	\begin{subfigure}{0.5\textwidth}
		\centering
		\includegraphics[width=\textwidth]{hybrid_image_scales.jpg}
		\caption{Two image combined scaled:\\ \href{hybrid_image_scales.jpg}{hybrid$\_$image$\_$scales.jpg}}
	\end{subfigure}
	%
\end{figure}


\end{document}
